\documentclass[11pt]{article}

\usepackage[margin=1in]{geometry}
\usepackage{enumitem}
\usepackage{hyperref}
\hypersetup{
    colorlinks=true,
    linkcolor=blue,
    urlcolor=blue
}

\setlist[itemize]{left=1.5em}

\begin{document}

\begin{center}
    \Large \textbf{Guidelines for Shorter Seminar Presentations} \\
    \normalsize Course Handout
\end{center}

\vspace{1em}

\section*{Purpose}
\begin{itemize}
    \item Practice clear, concise communication of technical or statistical material.
    \item Learn to structure a talk and highlight the most important contributions.
    \item Gain experience presenting to peers and handling questions.
\end{itemize}

\section*{Structure \& Timing}
\begin{itemize}
    \item \textbf{Introduction (1--2 min):} Motivate the problem, state the objective, outline the talk.
    \item \textbf{Main Content (6--7 min):} Present key methods, results, or arguments using 2--4 well-chosen visuals. Balance technical content (e.g., one key equation or framework) with applied or visual elements (e.g., a motivating example or figure).
    \item \textbf{Conclusion (1 min):} Summarize takeaways, limitations, or future directions.
    \item \textbf{Questions (1--2 min):} Expect 1--2 clarifying questions.
\end{itemize}

\textit{Tip: Rehearse with a timer at least once---students often underestimate how quickly 10 minutes passes.}

\section*{Presentation Guidelines}
\begin{itemize}
    \item \textbf{Slides (6--8 maximum):} Large fonts, minimal text, clear labeling, accessible colors and contrast, no clutter.
    \item \textbf{Delivery:}
    \begin{itemize}
        \item Speak clearly, maintain eye contact, avoid reading from slides.
        \item Use strategies to engage the audience (e.g., storytelling, pausing briefly to let key points sink in).
        \item Define technical terms when introduced.
        \item Practice pacing; presenters going over time will be stopped.
    \end{itemize}
    \item \textbf{Handling Questions \& Nerves:}
    \begin{itemize}
        \item Listen carefully, pause if needed, and respond concisely.
        \item It is fine to say ``I don’t know, but I can look into it.''
        \item Practicing with peers ahead of time can reduce nerves and improve confidence.
    \end{itemize}
\end{itemize}

\section*{Content Guidelines}
\begin{itemize}
    \item Assume the audience has general background in course material, but not topic specifics.
    \item Emphasize \textbf{big ideas, intuition, and implications} over excessive detail.
    \item Where relevant, include:
    \begin{itemize}
        \item A motivating example.
        \item A clear figure or simulation.
        \item One key equation or framework (not a page of math).
    \end{itemize}
\end{itemize}

\section*{Evaluation Criteria}
\begin{itemize}
    \item \textbf{Clarity:} Structure, pacing, effective use of visuals.
    \item \textbf{Understanding:} Accuracy and ability to answer questions.
    \item \textbf{Engagement:} Communication style, confidence, interaction with audience.
    \item \textbf{Conciseness:} Stays within time, emphasizes essentials.
\end{itemize}

\section*{Logistics}
\begin{itemize}
    \item \textbf{Timing:} 10 minutes total (8 min talk + 2 min questions).
    \item \textbf{Slides:} Email presentation to Professor Steorts by \textbf{10 AM on the day of class}.
    \item \textbf{Audience Role:} Each student should prepare one question or comment per talk.
    \item \textbf{Instructor Role:} Each student will receive feedback from me and should also receive feedback from faculty/peers.
\end{itemize}

\section*{Feedback}
At a minimum, please provide the speaker with the following:
\begin{enumerate}
    \item One thing you learned from their presentation.
    \item One strength (positive aspect of the presentation).
    \item One area for improvement (constructive, with an example).
    \item Any additional comments (both applied and technical).
\end{enumerate}

\textit{Tip: To make feedback more consistent, use the ``one strength, one improvement, one takeaway'' template.}

\end{document}

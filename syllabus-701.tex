\documentclass[11pt]{article}

\usepackage[margin=1in]{geometry}
\usepackage{enumitem}
\usepackage{hyperref}
\hypersetup{
    colorlinks=true,
    linkcolor=blue,
    urlcolor=blue
}

\begin{document}

\begin{center}
    \Large \textbf{STA 701: Readings in Statistical Science} \\
    \normalsize Duke University \\
    \vspace{0.5em}
    \textbf{Time:} Mondays, 11:45--1:00 \\
    \textbf{Place:} Old Chemistry 116 \\
    \textbf{Instructor:} Professor Rebecca Steorts
\end{center}

\vspace{1em}

\section*{Course Description}
This seminar provides graduate students with the opportunity to discuss their research with peers and faculty, while practicing effective research communication. Each session will consist of either two shorter presentations or one longer presentation. Presenters will receive constructive feedback from the instructor and are strongly encouraged to solicit feedback from peers.  

\section*{Course Goals}
By the end of this course, students will:
\begin{itemize}
    \item Develop skills for clear and concise oral communication of statistical research.
    \item Learn to structure presentations to highlight the most important contributions.
    \item Gain experience presenting to a scholarly audience and fielding questions.
    \item Provide and receive constructive feedback in a professional, supportive setting.
    \item Engage with peers’ research to broaden their understanding of statistical science.
\end{itemize}

\section*{Course Expectations}
Credit for this course is based on:
\begin{itemize}
    \item \textbf{Attendance:} Students are expected to attend all seminar sessions.
    \item \textbf{Participation:} Each student should actively engage by preparing at least one question or comment for every talk.
    \item \textbf{Speaker Responsibilities:} When scheduled to present, students are expected to prepare thoughtfully, deliver their presentation on time, and respond to feedback.
    \item \textbf{Introducing Speakers:} Students should be prepared to introduce their fellow speakers.
    \item \textbf{Discussion Leadership:} Students should help lead a short discussion on the topics presented, encouraging dialogue and questions.
    \item \textbf{Presentation Length:} Presentations will be assumed to follow the shorter 10-minute format unless students are explicitly notified otherwise.
    \item \textbf{Schedule Responsibility:} Students must check the posted schedule regularly. It is their responsibility to know when they are presenting and to be fully prepared.
\end{itemize}

\section*{Presentation Structure \& Timing}
\begin{itemize}
    \item \textbf{Introduction (1--2 min):} Motivate the problem, state the objective, outline the talk.
    \item \textbf{Main Content (6--7 min):} Present key methods, results, or arguments with 2--4 well-chosen visuals. Balance technical content (e.g., one key equation or framework) with applied or visual elements.
    \item \textbf{Conclusion (1 min):} Summarize takeaways, limitations, or future directions.
    \item \textbf{Questions (1--2 min):} Expect 1--2 clarifying questions.
\end{itemize}

\textit{Tip: Rehearse with a timer at least once---students often underestimate how quickly 10 minutes passes.}

\section*{Presentation Guidelines}
\begin{itemize}
    \item \textbf{Slides (6--8 maximum):} Large fonts, minimal text, clear labeling, accessible colors, no clutter.
    \item \textbf{Delivery:}
    \begin{itemize}
        \item Speak clearly, maintain eye contact, avoid reading from slides.
        \item Use strategies to engage the audience (e.g., storytelling, pausing to emphasize key points).
        \item Define technical terms when introduced.
        \item Presenters going over time will be stopped.
    \end{itemize}
    \item \textbf{Handling Questions \& Nerves:}
    \begin{itemize}
        \item Listen carefully, pause if needed, and respond concisely.
        \item It is fine to say ``I don’t know, but I can look into it.''
        \item Practicing with peers ahead of time can reduce nerves and improve confidence.
    \end{itemize}
\end{itemize}

\section*{Content Guidelines}
\begin{itemize}
    \item Assume the audience has general background in course material, but not topic specifics.
    \item Emphasize \textbf{big ideas, intuition, and implications} over excessive detail.
    \item Where relevant, include:
    \begin{itemize}
        \item A motivating example.
        \item A clear figure or simulation.
        \item One key equation or framework (not a page of math).
    \end{itemize}
\end{itemize}

\section*{Evaluation \& Feedback}
Presentations will be evaluated based on:
\begin{itemize}
    \item \textbf{Clarity:} Structure, pacing, effective use of visuals.
    \item \textbf{Understanding:} Accuracy and ability to answer questions.
    \item \textbf{Engagement:} Communication style, confidence, interaction with audience.
    \item \textbf{Conciseness:} Stays within time, emphasizes essentials.
\end{itemize}

\subsection*{Feedback Expectations}
\begin{itemize}
    \item Presenters will receive feedback from the instructor.  
    \item Audience members are strongly encouraged to provide feedback by email.  
    \item A minimal feedback template includes:
    \begin{enumerate}
        \item One strength (positive aspect of the presentation).
        \item One area for improvement (constructive, with an example).
        \item Any additional technical comments or suggestions.
    \end{enumerate}
\end{itemize}

\section*{Logistics}
\begin{itemize}
    \item \textbf{Timing:} Short presentations are 10 minutes total (8 min talk + 2 min questions). This shorter format will be assumed unless otherwise specified.
    \item \textbf{Slides:} Email your presentation to the instructor \textbf{24 hours before the scheduled talk}.
    \item \textbf{Title, Abstract, and Bio:} Email your title, abstract, and short bio (and webpage if you have one) to Professor Steorts as soon as possible.
    \item \textbf{Schedule Responsibility:} Students must monitor the schedule and ensure they are up to date on when they are presenting.
    \item \textbf{Audience Role:} Each student should prepare one question or comment per talk.
\end{itemize}

\end{document}
